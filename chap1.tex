\chapter{Magnetism}
\label{chapt1}

\section{Basic Magnetism}

The magnetic flux density, $\underline{B}$, measured in T within some material and in the presence of some external magnetic field, $\underline{H}$, is given by:

\begin{equation}
    \underline{B} = \mu_0 (\underline{H} + \underline{M}),
    \label{MagneticFluxDensity}
\end{equation}

\noindent where $\mu_0$ is the permeability of free space and $\underline{M}$ is the magnetisation of the material. Magnetisation is defined as the magnetic moment per unit volume.

\noindent If the magnetisation of the material is dependent on the external magnetic field such that:

\begin{equation}
    \underline{M} = \chi \underline{H},
    \label{LinearMagnetisationRelation}
\end{equation}

\noindent where $\chi$ is the magnetic susceptibility, then Eq~\ref{MagneticFluxDensity} may be rearranged as:

\begin{eqnarray}
    \underline{B} = \mu_0 (\underline{H} + \chi \underline{H}) \\
                  = \mu_0 (1 + \chi) \underline{H} \\
                  = \mu_0 \mu_r \underline{H}, \label{MagneticFluxDensityRelative}
\end{eqnarray}

\noindent where $\mu_r$ is the relative permeability. Eq~\ref{LinearMagnetisationRelation} is only valid in the case where the magnetisation of the material is linear in response to the application of an external magnetic field. This relation is not true for non-linear responses.

\noindent The value of $\chi$ can be used to classify magnetic materials. If $\chi < 0$, then the material is a diamagnet. Typically diamagnetic materials have $\chi \sim -10^{-5}$, although superconductors have $\chi = -1$ under certain conditions. Paramagnetic materials have $\chi > 0$ and $\sim 10^{-3}$. Ferromagnets have $\chi \gg 1$ with $\underline{M} \neq 0$, whilst anti-ferromagnets have $\chi > 0$ with $\underline{M} = 0$.

\noindent Materials that have full electron shells, such that there is no net electronic magnetic moment, are purely diamagnetic. Diamagnetism is always present when the material is subject to an external magnetic field but its effect is very small when compared to paramagnetic contributions. In comparison, materials are only paramagnetic if their atoms have partially filled shells.

\noindent The fundamental object in magnetism is the magnetic moment which is defined from classical electrodynamics as:

\begin{equation}
    \underline{\mu} = \int d \underline{\mu} = I \int d\underline{S},
    \label{MagneticMoment}
\end{equation}

\noindent where $I$ is the current flowing around a loop of area $dS$. This means that moving charges which form current loops have magnetic moments.

\noindent A massive object moving in a loop has angular momentum. So if this massive object also has a charge then it will produce a magnetic moment at the same time. These quantities are related via:

\begin{equation}
    \underline{\mu} = \gamma \underline{L}
    \label{MagneticMomentAngularMomentumRelationship},
\end{equation}

\noindent where $\gamma$ is the gyromagnetic ratio and $\underline{L}$ is the angular momentum. This relationship can be experimentally verified by the Einstein-de Haas effect or the Barnett effect, which rely on the conservation of angular momentum.

\noindent The energy of a magnetic moment, $E$, is given by:

\begin{equation}
    E = - \underline{\mu} \cdot \underline{B},
    \label{EnergyMagneticMoment}
\end{equation}

\noindent which is minimised when $\underline{\mu}$ is parallel to $\underline{B}$. From this we can see that the magnitude of the magnetisation of a material has the property that:

\begin{equation}
    M ~ \propto ~ \frac{\partial E}{\partial B}.
    \label{MagnetisationFromEnergy}
\end{equation}

\noindent The torque, $\underline{\tau}$, a magnetic moment feels due to a magnetic field is given by:

\begin{equation}
    \underline{\tau} = \underline{\mu} \times \underline{B}.
    \label{MagneticMomentTorque}
\end{equation}

\noindent This can be re-expressed using Eq~\ref{MagneticMomentAngularMomentumRelationship} and that torque is the rate of change of angular momentum, $\underline{\tau} = \frac{d \underline{L}}{dt}$:

\begin{equation}
    \frac{d\underline{\mu}}{dt} = \gamma \underline{\mu} \times \underline{B},
    \label{RateChangeMagneticMoment}
\end{equation}

\noindent which shows that $\underline{\mu}$ precesses around $\underline{B}$. The frequency associated with this precession is known as the Larmor frequency.

\noindent The Bohr magneton, $\mu_B$, is defined as:

\begin{equation}
    \mu_B = \frac{-e \hbar}{2m_e} \simeq 9.274 \times 10^{-24},
    \label{BohrMagneton}
\end{equation}

\noindent and represents a fundamental unit. It can be derived by considering an electron, in the ground state of an atom, forming a current loop of radius $r$ with area $\pi r^2$ and angular momentum of $L = m_e v r = \hbar$.

\noindent So far we have only considered isolated magnetic moments. We now briefly consider the effect of crystal electric fields, i.e the electric fields present in a material due to the surrounding ions and electrons, on electronic configurations and how this leads to orbital quenching.

\noindent For an isolated atom, the electronic s, p, d, f, ... orbitals each possess sub-orbitals that are degenerate. For example E($p_x$) = E($p_y$) = E($p_z$). However, this won't be the case when including the effect of the crystal electric field due to the coulombic attraction and repulsion occurring. The resulting energy change for each sub-orbital depends on the surrounding crystal structure and the corresponding overlap between sub-orbitals. This leads to the effect of orbital quenching. This is because the energy of these sub-orbitals are no longer degenerate and so, in the case of full quenching, $\langle L \rangle = 0$. This means experimental results of the effective magnetic moment of 3d electrons fit 2$\sqrt{S(S + 1)}$ better than $g_L\sqrt{J(J + 1)}$. The experimental results only match $g_L\sqrt{J(J + 1)}$ when $L = 0$, due to how electron sub-orbitals are filled according to Hund's rules.

\subsection{Key Results}

\section{Magnetic Interactions}

Ferromagnetic materials have a critical temperature, $T_C$, known as the Curie temperature. This represents the temperature at which the material undergoes a phase transition. The magnetic order observed for $T < T_C$ was originally thought to occur due to the magnetic dipolar interaction, which is given by:

\begin{equation}
    E = \frac{\mu_0}{4\pi r^3}[\underline{\mu}_1 \cdot \underline{\mu}_2 - \frac{3}{r^2}(\underline{\mu}_1 \cdot \underline{r})(\underline{\mu}_2 \cdot \underline{r})].
    \label{DipolarInteraction}
\end{equation}

\noindent However, this predicts critical temperatures much lower than those observed experimentally. So there must be another mechanism allowing long range magnetic order to occur. This is the exchange interaction, which relies on electrostatic interactions and the relative spin of the interacting fermions through the Pauli Exclusion Principle (PEP).

\noindent To see how this interaction arises we consider a system of two electrons with spins $\underline{S}_a$ and $\underline{S}_b$. Since electrons are fermions their overall wavefunctions must be anti-symmetric. The overall wavefunction is a product of a spatial and spin part:

\begin{equation}
    \Psi = \psi(\underline{r}) \chi(S),
    \label{OverallWavefunction}
\end{equation}

\noindent so either $\psi(\underline{r}) = \psi(-\underline{r})$ and $\chi(S) = - \chi(S)$ or $\psi(\underline{r}) = -\psi(-\underline{r})$ and $\chi(S) =  \chi(S)$.

\noindent The total vector spin of our two electron system is:

\begin{equation}
    \underline{S}_{tot} = \underline{S}_a + \underline{S}_b,
    \label{TotalVectorSpin}
\end{equation}

\noindent which means that:

\begin{equation}
    |\underline{S}_{tot}|^2 = |\underline{S}_a|^2 + |\underline{S}_b|^2 + 2 \underline{S}_a \cdot \underline{S}_b = s(s + 1),
    \label{TotalSpin}
\end{equation}

\noindent where we have let $\hbar = 1$. For the combined system we may either have a singlet state, $s = 0$, or a triplet state, $s = 1$, which means that $|\underline{S}_{tot}|^2 = 0$ or $|\underline{S}_{tot}|^2 = 2$, using Eq~\ref{TotalSpin}. From this we may calculate $\underline{S}_a \cdot \underline{S}_b$ in Eq~\ref{TotalSpin} by recalling that $|\underline{S}_a|^2 = |\underline{S}_b|^2$. As a result of this for $s = 0$, $\underline{S}_a \cdot \underline{S}_b = \frac{-3}{4}$ and for $s = 1$, $\underline{S}_a \cdot \underline{S}_b = \frac{1}{4}$.

\noindent The possible eigenstate basis of the system is formed from the projection of the two spins along the z axis, i.e: $\ket{\uparrow \uparrow}$, $\ket{\downarrow \downarrow}$, $\ket{\uparrow \downarrow}$ and $\ket{\downarrow \uparrow}$. However, by themselves the last two are not symmetric or anti-symmetric as we require them so we form linear combinations to satisfy this requirement. In conclusion:

\begin{center}
\begin{tabular}{||c c c c c c||} 
 \hline
 Eigenstate & $m_s$ & $s$ & $\underline{S}_a \cdot \underline{S}_b$ & $\chi(S)$ & $\psi(\underline{r})$ \\ [0.5ex] 
 \hline\hline
 $\ket{\uparrow \uparrow}$ & 1 & 1 & $\frac{1}{4}$ & Symmetric & Anti-symmetric \\ 
 \hline
 $\frac{\ket{\uparrow \downarrow} + \ket{\downarrow \uparrow}}{\sqrt{2}}$ & 0 & 1 & $\frac{1}{4}$ & Symmetric & Anti-symmetric \\
 \hline
 $\ket{\downarrow \downarrow}$ & -1 & 1 & $\frac{1}{4}$ & Symmetric & Anti-symmetric \\
 \hline
 $\frac{\ket{\uparrow \downarrow} - \ket{\downarrow \uparrow}}{\sqrt{2}}$ & 0 & 0 & $\frac{-3}{4}$ & Anti-symmetric & Symmetric \\
 \hline
\end{tabular}
\end{center}

\noindent The first three eigenstates correspond to the triplet state and have spin wavefunction, $\chi_T$, whilst the final eigenstate corresponds to the singlet state and has spin wavefunction, $\chi_S$.

\noindent From the table we can see that Eq~\ref{OverallWavefunction} has the following form for singlet states:

\begin{equation}
    \Psi_S = \frac{1}{\sqrt{2}} [\psi_a(\underline{r}_1) \psi_b(\underline{r}_2) + \psi_a(\underline{r}_2) \psi_b(\underline{r}_1)] \chi_S,
    \label{SingletStateWavefunction}
\end{equation}

\noindent and for the triplet state:

\begin{equation}
    \Psi_T = \frac{1}{\sqrt{2}} [\psi_a(\underline{r}_1) \psi_b(\underline{r}_2) - \psi_a(\underline{r}_2) \psi_b(\underline{r}_1)] \chi_T.
    \label{TripletStateWavefunction}
\end{equation}

\noindent The corresponding energies of these states are $E_S = \bra{\Psi_S} \Hat{H} \ket{\Psi_S}$ and $E_T = \bra{\Psi_T} \Hat{H} \ket{\Psi_T}$ respectively. This means we can now define the exchange constant (or integral) as:

\begin{equation}
    J = \frac{E_S - E_T}{2},
    \label{ExchangeConstant}
\end{equation}

\noindent which is used in a two body Hamiltonian:

\begin{equation}
    H = -2J\underline{S}_a \cdot \underline{S}_b.
    \label{TwoBodyHamiltonian}
\end{equation}

\noindent This can then be generalised to a many body system with the Heisenberg Hamiltonian:

\begin{equation}
    H = -\Sigma_{i,j} J_{ij} \underline{S}_a \cdot \underline{S}_b = -2 \Sigma_{i>j} J_{ij} \underline{S}_a \cdot \underline{S}_b
    \label{HeisenbergHamiltonian},
\end{equation}

\noindent where if $J>0$, the triplet state is favoured and the material will be ferromagnetic but if $J<0$ then the singlet state is favoured and the material will be anti-ferromagnetic.

\noindent If the electrons are on the same atom $J>0$, meaning they form an anti-symmetric spatial state which minimises the coulombic repulsion between them by the keeping the electrons apart. This is consistent with Hund's 1st rule.

\noindent However if the two electrons are on neighbouring atoms we now consider molecular instead of atomic orbitals, since the energy of an electron in 1D potential well scales as $L^{-2}$, where $L$ is the size of the well. These molecular orbitals are either bonding (spatially symmetric) or anti-bonding (spatially anti-symmetric). In this case, because the anti-bonding orbital isn't the lowest energy state, we must have $J<0$ and so singlet states are preferred. This exchange interaction is between electrons on neighbouring magnetic atoms and so is known as direct exchange. This is physically rare.

\noindent We now consider some examples of indirect exchange which is more common.
The first is superexchange which occurs in ionic solids such as MnO: Mn$^{2+}$O$^{2-}$. Here the exchange interaction is mediated between the magnetic Mn$^{2+}$ ions by the non-magnetic O$^{2-}$ ion which separates the magnetic ions. By considering the excited states of the spins on the ions of such a system we find that indirect exchange results in anti-ferromagnetism.
In metals, the exchange interaction between the magnetic ions is mediated by the conduction electrons. This is known as the RKKY interaction:

\begin{equation}
    J_{RKKY}(r) \propto \frac{cos(2k_fr)}{r^3},
    \label{IndirectInteractionRKKY}
\end{equation}

\noindent and describes magnetic order in metals with no direct exchange, superexchange or band magnetism. The Stoner model explains band magnetims which also occurs in metals.
Another example of indirect exchange is double exchange. This is a ferromagnetic interaction which occurs when magnetic ions within a material have mixed valency, such as Mn$^{3+}$ and Mn$^{4+}$. In this case t$_{2g}$ and e$_g$ orbitals are considered and we find that anti-ferromagnetic interactions violates Hund's 1st rule and leads to higher energy configurations than ferromagnetic behaviour.

\noindent So far all these interactions have been isotropic. An example  of an anisotropic interaction is the DM interaction, which has the Hamiltonian:

\begin{equation}
    H_{DM} = \underline{D} \cdot \underline{S}_a \times \underline{S}_b.
    \label{AnisotropicInteractionDM}
\end{equation}

\noindent In this case, the energy of the system is minimised when magnetic moments are perpendicular. If this interaction is present in an anti-ferromagnetic system then the system becomes weakly ferromagnetic. This interaction is anisotropic because it occurs between the ground state of one magnetic ion and the excited state of another.

\noindent Most materials possess some magnetic anisotropy. This means it is energetically favourable for the material to be magnetised along a particular direction, which normally arises due to the crystal structure of the material. This means:

\begin{equation}
    E \propto \Sigma_i (S^Z_i)^2,
    \label{MagneticAnisotropyEnergy}
\end{equation}

\noindent where $D$ represents some characteristic property of the system providing anisotropy. If $D < 0$ then we have whats known as the easy axis along the z axis, whilst if $D > 0$ then the easy plane corresponds to the xy plane. This concept leads to the Ising Hamiltonian:

\begin{equation}
    H = -\Sigma_{i,j} J_{ij} S^Z_i S^Z_j,
    \label{IsingHamiltonian}
\end{equation}

\noindent and the XY hamiltonian:

\begin{equation}
    H = -\Sigma_{i,j} J_{ij} (S^X_i S^X_j + S^Y_i S^Y_j),
    \label{HamiltonianXY}
\end{equation}

\noindent respectively.

\section{Magnetic Measurement Techniques}

NEED IMAGES FOR EXPERIMENTAL SETUPS AND RESULTS.

\subsection{Magnetic Resonance}

The first technique is that of magnetic resonance. The experimental setup is shown below:

\begin{figure}
    \centering
    \includegraphics{}
    \caption{Caption}
    \label{fig:enter-label}
\end{figure}

\noindent Normally the frequency of microwave, $\nu$, is kept constant whilst the magnetic field is sweeped. The absorption of microwaves by the sample is determined by monitoring the Q factor of the cavity. In such magnetic dipole transitions the selection rules are $\Delta m_J = \pm1$ and $\Delta m_I = 0$.

\noindent For a $J = 1$ ion, such as Ni$^{2+}$m with no crystal field splitting we observe:

\begin{figure}
    \centering
    \includegraphics{}
    \caption{Caption}
    \label{fig:enter-label}
\end{figure}

\noindent However, with crystal field splitting:

\begin{figure}
    \centering
    \includegraphics{}
    \caption{Caption}
    \label{fig:enter-label}
\end{figure}

\noindent This allows us to determine the crystal field splitting, $\Delta$. A more complete picture, for Ce$^{3+}$, is shown below:

\begin{figure}
    \centering
    \includegraphics{}
    \caption{Caption}
    \label{fig:enter-label}
\end{figure}

\subsection{Magnetic Susceptibility Measurements}

We now consider two techniques that allow magnetic susceptibility measurements. The first is that of a torque magnetometer, as shown below:

\begin{figure}
    \centering
    \includegraphics{}
    \caption{Caption}
    \label{fig:enter-label}
\end{figure}

\noindent A torque, $\underline{\tau} = \underline{\mu} \times \underline{B}$, is produced on the sample using the solenoids which causes the wire to rotate. This rotation is detected by the photodetectors which then supply a current to the compensation coil which produces an opposing torque using the permanent magnets of known strength. The current is varied in order to compensate for the torque of the sample. These magnetometers are very sensitive but don't give the absolute value for the torque/magnetisation.

\noindent These can be measured exactly using extraction magnetometers:

\begin{figure}
    \centering
    \includegraphics{}
    \caption{Caption}
    \label{fig:enter-label}
\end{figure}

\noindent Here a magnetic material oscillates between a coil of wire, which through Lenz's law produces a measurable voltage:

\begin{equation}
    \epsilon = - \frac{d \Phi}{dt}
    \label{LenzLaw}
\end{equation}

\noindent This is the basic process behind vibrating sample magnetometers (VSMs) and superconducting quantum interference devices (SQUIDs).

\subsection{Calorimetry}

These type of measurements allow the determination of the heat capacity of a material which possesses knowledge of the interactions that have determined its properties. It also allows for the investigation of phase transitions.

\noindent The magnetic entropy of a system can be estimated from its heat capacity:

\begin{equation}
    S(T_f) = \int_0^{T_f} \frac{C(T)}{T} dT,
    \label{EstimatedMagneticEntropy}
\end{equation}

\noindent and then compared to the expected entropy of a magnetic ion with total angular momentum $J$:

\begin{equation}
    S = Rln(2J + 1)
    \label{ExactMagneticEntropy}.
\end{equation}

\noindent The principal method for determining the specific heat capacity is that of thermal relaxation. This measures the temperature of a material as a function of time and heater power. From the time constant associated with the heating and cooling the specific heat capacity can be determined.

\subsection{Magnetic Neutron Scattering}

This is typically performed using a triple-axis spectrometer. The monochromator crystal ensures that the neutrons have a single wavelength and not a distribution as they would have from a source.

\noindent The neutron has some properties that make it useful:

\begin{itemize}
    \item Similar wavelength to atomic spacings which means that diffraction measurements may be performed.
    \item Similar energy to atomic/electronic processes which allows for spectroscopy measurements.
    \item They are highly penetrating and so can be used for bulk measurements.
    \item They are non-destructive (when slow enough) so can be used to probe delicate samples.
\end{itemize}

\noindent Elastic magnetic scattering probes static magnetic structures whilst inelastic scattering probes the spin dynamics. The magnetic scattering cross section is proportional to the fourier transform of the spin correlation function.

\subsection{M\"ossbauer Spectroscopy}

This relies on the radioactive decay of $57^$Co to an excited state of $57^$Fe.

\begin{figure}
    \centering
    \includegraphics{}
    \caption{Caption}
    \label{fig:enter-label}
\end{figure}

\noindent The gamma ray emitted from the $57^$Fe between $I = \frac{3}{2}$ and $I = \frac{1}{2}$ has an energy of 14.4~keV. It is then directed towards a sample containing $57^$Fe. The doppler effect is present as the $57^$Co source moves towards the sample. The resulting absorption of photons is detected as a function of the speed fo the $57^$Co source. There is no recoil affecting the relative velocity due to the source being a bulk material. There is also a small probability of emission/absorption without a phonon. This process can be used to investigate quadrapole or magnetic splitting. However this process only works for certain materials.

\subsection{Muon-Spin Rotation}

Muons are prepared by colliding high energy protons with carbon nuclei which produces pions which in turn decay:

\begin{equation}
    \pi^+ \rightarrow \mu^+ + \nu_\mu
    \label{PionDecay}
\end{equation}

\noindent These muons emerge spin polarised and are implanted into the sample. Muons have a relatively long lifetime but eventually decay into positrons:

\begin{equation}
    \mu^+ \rightarrow e^+ + \Bar{\nu}_\mu + \nu_e
    \label{MuonDecay}
\end{equation}

\noindent These positrons tend to leave the sample along the direction which the muon spins was aligned along when the muon decayed. Typically this follows the distribution:

\begin{equation}
    W(\theta) = 1 + \alpha cos(\theta)
    \label{PositronDistribution}
\end{equation}

\noindent This method probes the local magnetic field of where the muon was implanted.

\section{Magnetic Domains}

When ferromagnets are cooled to $T < T_C$, magnetic domains are formed. These are regions of uniform magnetisation and are separated by domain walls of finite size.
The main domain walls are N\'eel walls and Bloch walls. In N\'eel walls the plane of rotation of magnetic moments is perpendicular to the plane of the wall, whilst for Bloch walls it is parallel.

\begin{figure}
    \centering
    \includegraphics{}
    \caption{Caption}
    \label{fig:enter-label}
\end{figure}

\noindent We now explore the Bloch wall further. The exchange energy of two spins in the wall is:

\begin{equation}
    E = -2J\underline{S}_1 \cdot \underline{S}_2 = -2JS^2\cos(\theta) \approx -2JS^2(1 - \frac{\theta^2}{2})
    \label{ExchangeEnergy2SpinBlochWall}
\end{equation}

\noindent This means that the energy cost to rotate one spin relative to another is given by:

\begin{equation}
    JS^2\theta^2
    \label{EnergyCostRotationBlochWall}
\end{equation}

\noindent So the energy cost for a rotation by $\pi$ with $N$ spins is, using Eq~\ref{EnergyCostRotationBlochWall}, given by:

\begin{equation}
    JS^2(\frac{\pi}{N})^2 N = \frac{JS^2\pi^2}{N},
    \label{FullEnergyCostBlochWall}
\end{equation}

\noindent which means the energy per unit area of a Bloch wall, $\sigma_{BW}$, in a cubic crystal is given by:

\begin{equation}
    \sigma_{BW} =  \frac{JS^2\pi^2}{Na^2},
    \label{EnergyPerUnitAreaBlochWall}
\end{equation}

\noindent where $a$ is the lattice parameter. However, this implies that $\sigma_{BW} \rightarrow 0$ as $N \rightarrow \infty$, meaning no domain walls would form. The interaction that prevents this and allows the formation of domain walls is due to magnetocrystalline anisotropy.

\noindent The energy cost comes from when the spins in the domain wall rotate. This means that they have a component along the hard plane/axis which in turn costs energy. We can model this magnetically preferred direction with a magnetic anisotropy energy given by:

\begin{equation}
    E = D\sin^2(\theta),
    \label{AnisotropyEnergy}
\end{equation}

\noindent where $D > 0$ is the anisotropy constant. So summing over all magnetic moments:

\begin{equation}
    \Sigma_{i = 1}^N D\sin^2(\theta) \approx \frac{N}{\pi} \int_0^\pi D\sin^2(\theta) d\theta = \frac{ND}{2}
    \label{TotalAnisotropyEnergyDensity}
\end{equation}

\noindent This means that the Bloch wall energy per unit area, combining Eq~\ref{EnergyPerUnitAreaBlochWall} and Eq~\ref{TotalAnisotropyEnergyDensity}, is given by:

\begin{equation}
    \sigma_{BW} = \frac{JS^2\pi^2}{Na^2} + \frac{NDa}{2}
    \label{FinalEnergyUnitAreaBlochWallExpression}
\end{equation}

\noindent From which we can find the minimum number of spins comprising the wall:

\begin{equation}
    \frac{d \sigma_{BW}}{dN} = 0 \Rightarrow N = \pi S \sqrt{\frac{2J}{Da^3}
    }
    \label{NumberSpinsBlochWall}
\end{equation}

\noindent This in turn means that the width of the wall is:

\begin{equation}
    \delta = Na = \pi S \sqrt{\frac{2J}{Da}},
    \label{BlochWallWidth}
\end{equation}

\noindent and also that:

\begin{equation}
    \sigma_{BW} = \pi S \sqrt{\frac{2JD}{a}}
    \label{FinalEnergyUnitAreaBlochWall}
\end{equation}

\noindent The formation of domain walls helps save dipolar energy. This is because of the necessity to satisfy $\underline{\nabla} \cdot \underline{B}$. So whenever a system possesses a magnetisation it also has a demagnetising field since, using Eq~\ref{MagneticFluxDensity}, $\underline{\nabla} \cdot \underline{M} = - \underline{\nabla} \cdot \underline{H}$.

\noindent The (surface) demagnetisation energy is given by:

\begin{equation}
    E = \frac{-\mu_0}{2} \int_V \underline{M} \cdot \underline{H}_d d\tau
    \label{DemagnetisationEnergy}
\end{equation}

\noindent For an ellipsoidal shape, magnetised along one of its principal axes:

\begin{equation}
    \underline{H}_d = - N\underline{M},
    \label{DemagnetisingField}
\end{equation}

\noindent where $N$ is the demagnetising factor. Combining Eq~\ref{DemagnetisationEnergy} and Eq~\ref{DemagnetisingField} we find that:

\begin{equation}
    E = \frac{\mu_0 N M^2 V}{2}
    \label{TotalDemagnetisationEnergy}
\end{equation}

\noindent This dipolar energy may be saved by breaking the sameple into domains. However the number of domains formed is a balance between the cost of demagnetisation and the cost of domain walls.

\begin{figure}
    \centering
    \includegraphics{}
    \caption{Caption}
    \label{fig:enter-label}
\end{figure}

\noindent For $\underline{H}$ parallel to a long cylindrical rod $N = 0$, whilst $N = 1$ for $\underline{H}$ perpendicular to a flat plate and $N = \frac{1}{3}$ for a sphere. Although $N$ is actually a matrix (trace must equal 1?).

\noindent The magnetisation of a material as a function of applied field in the form of a hysteresis loop shows the behaviour due to domains.

\begin{figure}
    \centering
    \includegraphics{}
    \caption{Caption}
    \label{fig:enter-label}
\end{figure}

\noindent This behaviour is due to the domain walls becoming pinned on defects. The magnetisation rotation region is where the domains are rotated away from the easy axis/plane due to the strength of the magnetic field.

\noindent Hysteresis loops can distinguish between hard and soft magnets:

\begin{figure}
    \centering
    \includegraphics{}
    \caption{Caption}
    \label{fig:enter-label}
\end{figure}

\noindent Soft magnets have the following properties:

\begin{itemize}
    \item Small anisotropy energy.
    \item Domain walls are broad and easy to move
    \item Small coercive fields
\end{itemize}

\noindent whereas hard magnets:

\begin{itemize}
    \item Domains are narrow and hard to move due to large anisotropy energy.
    \item No spontaneous magnetisation.
\end{itemize}

Magnetic materials can undergo a process known as magnetostriction. This is where the shape of a magnetic material changes when magnetised. This occurs since the crystal can lower its anisotropy energy by deforming itself. This is dependent on strain.

\noindent We now consider single domain small magnetic particles:

\begin{figure}
    \centering
    \includegraphics{}
    \caption{Caption}
    \label{fig:enter-label}
\end{figure}

\noindent It is possible for small magnetic particles with a single domain to form.

\noindent The behaviour of such a particle in an external magnetic field can be investigated with:

\begin{equation}
    E = D\sin^2(\theta - \phi) - \mu_0 H M \cos(\phi),
    \label{SingleDomainMagneticParticleEnergy}
\end{equation}

\noindent where $\theta$ is the angle between $\underline{H}$ and the easy axis, and $\phi$ is the angle between $\underline{M}$ and $\underline{H}$. If $\theta = 90^\circ$ we get a hysteresis loop for a soft magnet with plateaus at extreme magnetic fields and if $\theta = 0^\circ$ we obtain the hysteresis loop for a hard magnet.

\section{Paramagnets and Ferromagnets}

\section{Landau Theory}

\section{Anti-Ferromagnets}

\section{More Magnetic Order Types}

\section{Spin Waves and Magnons}