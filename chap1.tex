\chapter{Magnetism}
\label{chapt1}

\section{Basic Magnetism}

The magnetic flux density, $\underline{B}$, measured in T within some material and in the presence of some external magnetic field, $\underline{H}$, is given by:

\begin{equation}
    \underline{B} = \mu_0 (\underline{H} + \underline{M}),
    \label{MagneticFluxDensity}
\end{equation}

\noindent where $\mu_0$ is the permeability of free space and $\underline{M}$ is the magnetisation of the material. Magnetisation is defined as the magnetic moment per unit volume.

\noindent If the magnetisation of the material is dependent on the external magnetic field such that:

\begin{equation}
    \underline{M} = \chi \underline{H},
    \label{LinearMagnetisationRelation}
\end{equation}

\noindent where $\chi$ is the magnetic susceptibility, then Eq~\ref{MagneticFluxDensity} may be rearranged as:

\begin{eqnarray}
    \underline{B} = \mu_0 (\underline{H} + \chi \underline{H}) \\
                  = \mu_0 (1 + \chi) \underline{H} \\
                  = \mu_0 \mu_r \underline{H}, \label{MagneticFluxDensityRelative}
\end{eqnarray}

\noindent where $\mu_r$ is the relative permeability. Eq~\ref{LinearMagnetisationRelation} is only valid in the case where the magnetisation of the material is linear in response to the application of an external magnetic field. This relation is not true for non-linear responses.

\noindent The value of $\chi$ can be used to classify magnetic materials. If $\chi < 0$, then the material is a diamagnet. Typically diamagnetic materials have $\chi \sim -10^{-5}$, although superconductors have $\chi = -1$ under certain conditions. Paramagnetic materials have $\chi > 0$ and $\sim 10^{-3}$. Ferromagnets have $\chi \gg 1$ with $\underline{M} \neq 0$, whilst anti-ferromagnets have $\chi > 0$ with $\underline{M} = 0$.

\noindent Materials that have full electron shells, such that there is no net electronic magnetic moment, are purely diamagnetic. Diamagnetism is always present when the material is subject to an external magnetic field but its effect is very small when compared to paramagnetic contributions. In comparison, materials are only paramagnetic if their atoms have partially filled shells.

\noindent The fundamental object in magnetism is the magnetic moment which is defined from classical electrodynamics as:

\begin{equation}
    \underline{\mu} = \int d \underline{\mu} = I \int d\underline{S},
    \label{MagneticMoment}
\end{equation}

\noindent where $I$ is the current flowing around a loop of area $dS$. This means that moving charges which form current loops have magnetic moments.

\noindent A massive object moving in a loop has angular momentum. So if this massive object also has a charge then it will produce a magnetic moment at the same time. These quantities are related via:

\begin{equation}
    \underline{\mu} = \gamma \underline{L}
    \label{MagneticMomentAngularMomentumRelationship},
\end{equation}

\noindent where $\gamma$ is the gyromagnetic ratio and $\underline{L}$ is the angular momentum. This relationship can be experimentally verified by the Einstein-de Haas effect or the Barnett effect, which rely on the conservation of angular momentum.

\noindent The energy of a magnetic moment, $E$, is given by:

\begin{equation}
    E = - \underline{\mu} \cdot \underline{B},
    \label{EnergyMagneticMoment}
\end{equation}

\noindent which is minimised when $\underline{\mu}$ is parallel to $\underline{B}$. From this we can see that the magnitude of the magnetisation of a material has the property that:

\begin{equation}
    M ~ \propto ~ \frac{\partial E}{\partial B}.
    \label{MagnetisationFromEnergy}
\end{equation}

\noindent The torque, $\underline{\tau}$, a magnetic moment feels due to a magnetic field is given by:

\begin{equation}
    \underline{\tau} = \underline{\mu} \times \underline{B}.
    \label{MagneticMomentTorque}
\end{equation}

\noindent This can be re-expressed using Eq~\ref{MagneticMomentAngularMomentumRelationship} and that torque is the rate of change of angular momentum, $\underline{\tau} = \frac{d \underline{L}}{dt}$:

\begin{equation}
    \frac{d\underline{\mu}}{dt} = \gamma \underline{\mu} \times \underline{B},
    \label{RateChangeMagneticMoment}
\end{equation}

\noindent which shows that $\underline{\mu}$ precesses around $\underline{B}$. The frequency associated with this precession is known as the Larmor frequency.

\noindent The Bohr magneton, $\mu_B$, is defined as:

\begin{equation}
    \mu_B = \frac{-e \hbar}{2m_e} \simeq 9.274 \times 10^{-24},
    \label{BohrMagneton}
\end{equation}

\noindent and represents a fundamental unit. It can be derived by considering an electron, in the ground state of an atom, forming a current loop of radius $r$ with area $\pi r^2$ and angular momentum of $L = m_e v r = \hbar$.

\noindent So far we have only considered isolated magnetic moments. We now briefly consider the effect of crystal electric fields, i.e the electric fields present in a material due to the surrounding ions and electrons, on electronic configurations and how this leads to orbital quenching.

\noindent For an isolated atom, the electronic s, p, d, f, ... orbitals each possess sub-orbitals that are degenerate. For example E($p_x$) = E($p_y$) = E($p_z$). However, this won't be the case when including the effect of the crystal electric field due to the coulombic attraction and repulsion occurring. The resulting energy change for each sub-orbital depends on the surrounding crystal structure and the corresponding overlap between sub-orbitals. This leads to the effect of orbital quenching. This is because the energy of these sub-orbitals are no longer degenerate and so, in the case of full quenching, $\langle L \rangle = 0$. This means experimental results of the effective magnetic moment of 3d electrons fit 2$\sqrt{S(S + 1)}$ better than $g_L\sqrt{J(J + 1)}$. The experimental results only match $g_L\sqrt{J(J + 1)}$ when $L = 0$, due to how electron sub-orbitals are filled according to Hund's rules.

\subsection{Key Results}

\section{Magnetic Interactions}

Ferromagnetic materials have a critical temperature, $T_C$, known as the Curie temperature. This represents the temperature at which the material undergoes a phase transition. The magnetic order observed for $T < T_C$ was originally thought to occur due to the magnetic dipolar interaction, which is given by:

\begin{equation}
    E = \frac{\mu_0}{4\pi r^3}[\underline{\mu}_1 \cdot \underline{\mu}_2 - \frac{3}{r^2}(\underline{\mu}_1 \cdot \underline{r})(\underline{\mu}_2 \cdot \underline{r})].
    \label{DipolarInteraction}
\end{equation}

\noindent However, this predicts critical temperatures much lower than those observed experimentally. So there must be another mechanism allowing long range magnetic order to occur. This is the exchange interaction, which relies on electrostatic interactions and the relative spin of the interacting fermions through the Pauli Exclusion Principle (PEP).

\noindent To see how this interaction arises we consider a system of two electrons with spins $\underline{S}_a$ and $\underline{S}_b$. Since electrons are fermions their overall wavefunctions must be anti-symmetric. The overall wavefunction is a product of a spatial and spin part:

\begin{equation}
    \Psi = \psi(\underline{r}) \chi(S),
    \label{OverallWavefunction}
\end{equation}

\noindent so either $\psi(\underline{r}) = \psi(-\underline{r})$ and $\chi(S) = - \chi(S)$ or $\psi(\underline{r}) = -\psi(-\underline{r})$ and $\chi(S) =  \chi(S)$.

\noindent The total vector spin of our two electron system is:

\begin{equation}
    \underline{S}_{tot} = \underline{S}_a + \underline{S}_b,
    \label{TotalVectorSpin}
\end{equation}

\noindent which means that:

\begin{equation}
    |\underline{S}_{tot}|^2 = |\underline{S}_a|^2 + |\underline{S}_b|^2 + 2 \underline{S}_a \cdot \underline{S}_b = s(s + 1),
    \label{TotalSpin}
\end{equation}

\noindent where we have let $\hbar = 1$. For the combined system we may either have a singlet state, $s = 0$, or a triplet state, $s = 1$, which means that $|\underline{S}_{tot}|^2 = 0$ or $|\underline{S}_{tot}|^2 = 2$, using Eq~\ref{TotalSpin}. From this we may calculate $\underline{S}_a \cdot \underline{S}_b$ in Eq~\ref{TotalSpin} by recalling that $|\underline{S}_a|^2 = |\underline{S}_b|^2$. As a result of this for $s = 0$, $\underline{S}_a \cdot \underline{S}_b = \frac{-3}{4}$ and for $s = 1$, $\underline{S}_a \cdot \underline{S}_b = \frac{1}{4}$.

\noindent The possible eigenstate basis of the system is formed from the projection of the two spins along the z axis, i.e: $\ket{\uparrow \uparrow}$, $\ket{\downarrow \downarrow}$, $\ket{\uparrow \downarrow}$ and $\ket{\downarrow \uparrow}$. However, by themselves the last two are not symmetric or anti-symmetric as we require them so we form linear combinations to satisfy this requirement. In conclusion:

\begin{center}
\begin{tabular}{||c c c c c c||} 
 \hline
 Eigenstate & $m_s$ & $s$ & $\underline{S}_a \cdot \underline{S}_b$ & $\chi(S)$ & $\psi(\underline{r})$ \\ [0.5ex] 
 \hline\hline
 $\ket{\uparrow \uparrow}$ & 1 & 1 & $\frac{1}{4}$ & Symmetric & Anti-symmetric \\ 
 \hline
 $\frac{\ket{\uparrow \downarrow} + \ket{\downarrow \uparrow}}{\sqrt{2}}$ & 0 & 1 & $\frac{1}{4}$ & Symmetric & Anti-symmetric \\
 \hline
 $\ket{\downarrow \downarrow}$ & -1 & 1 & $\frac{1}{4}$ & Symmetric & Anti-symmetric \\
 \hline
 $\frac{\ket{\uparrow \downarrow} - \ket{\downarrow \uparrow}}{\sqrt{2}}$ & 0 & 0 & $\frac{-3}{4}$ & Anti-symmetric & Symmetric \\
 \hline
\end{tabular}
\end{center}

\noindent The first three eigenstates correspond to the triplet state and have spin wavefunction, $\chi_T$, whilst the final eigenstate corresponds to the singlet state and has spin wavefunction, $\chi_S$.

\noindent From the table we can see that Eq~\ref{OverallWavefunction} has the following form for singlet states:

\begin{equation}
    \Psi_S = \frac{1}{\sqrt{2}} [\psi_a(\underline{r}_1) \psi_b(\underline{r}_2) + \psi_a(\underline{r}_2) \psi_b(\underline{r}_1)] \chi_S,
    \label{SingletStateWavefunction}
\end{equation}

\noindent and for the triplet state:

\begin{equation}
    \Psi_T = \frac{1}{\sqrt{2}} [\psi_a(\underline{r}_1) \psi_b(\underline{r}_2) - \psi_a(\underline{r}_2) \psi_b(\underline{r}_1)] \chi_T.
    \label{TripletStateWavefunction}
\end{equation}

\noindent The corresponding energies of these states are $E_S = \bra{\Psi_S} \Hat{H} \ket{\Psi_S}$ and $E_T = \bra{\Psi_T} \Hat{H} \ket{\Psi_T}$ respectively. This means we can now define the exchange constant (or integral) as:

\begin{equation}
    J = \frac{E_S - E_T}{2},
    \label{ExchangeConstant}
\end{equation}

\noindent which is used in a two body Hamiltonian:

\begin{equation}
    H = -2J\underline{S}_a \cdot \underline{S}_b.
    \label{TwoBodyHamiltonian}
\end{equation}

\noindent This can then be generalised to a many body system with the Heisenberg Hamiltonian:

\begin{equation}
    H = -\Sigma_{i,j} J_{ij} \underline{S}_a \cdot \underline{S}_b = -2 \Sigma_{i>j} J_{ij} \underline{S}_a \cdot \underline{S}_b
    \label{HeisenbergHamiltonian},
\end{equation}

\noindent where if $J>0$, the triplet state is favoured and the material will be ferromagnetic but if $J<0$ then the singlet state is favoured and the material will be anti-ferromagnetic.

\noindent If the electrons are on the same atom $J>0$, meaning they form an anti-symmetric spatial state which minimises the coulombic repulsion between them by the keeping the electrons apart. This is consistent with Hund's 1st rule.

\noindent However if the two electrons are on neighbouring atoms we now consider molecular instead of atomic orbitals, since the energy of an electron in 1D potential well scales as $L^{-2}$, where $L$ is the size of the well. These molecular orbitals are either bonding (spatially symmetric) or anti-bonding (spatially anti-symmetric). In this case, because the anti-bonding orbital isn't the lowest energy state, we must have $J<0$ and so singlet states are preferred. This exchange interaction is between electrons on neighbouring magnetic atoms and so is known as direct exchange. This is physically rare.

\noindent We now consider some examples of indirect exchange which is more common.
The first is superexchange which occurs in ionic solids such as MnO: Mn$^{2+}$O$^{2-}$. Here the exchange interaction is mediated between the magnetic Mn$^{2+}$ ions by the non-magnetic O$^{2-}$ ion which separates the magnetic ions. By considering the excited states of the spins on the ions of such a system we find that indirect exchange results in anti-ferromagnetism.
In metals, the exchange interaction between the magnetic ions is mediated by the conduction electrons. This is known as the RKKY interaction:

\begin{equation}
    J_{RKKY}(r) \propto \frac{cos(2k_fr)}{r^3},
    \label{IndirectInteractionRKKY}
\end{equation}

\noindent and describes magnetic order in metals with no direct exchange, superexchange or band magnetism. The Stoner model explains band magnetims which also occurs in metals.
Another example of indirect exchange is double exchange. This is a ferromagnetic interaction which occurs when magnetic ions within a material have mixed valency, such as Mn$^{3+}$ and Mn$^{4+}$. In this case t$_{2g}$ and e$_g$ orbitals are considered and we find that anti-ferromagnetic interactions violates Hund's 1st rule and leads to higher energy configurations than ferromagnetic behaviour.

\noindent So far all these interactions have been isotropic. An example  of an anisotropic interaction is the DM interaction, which has the Hamiltonian:

\begin{equation}
    H_{DM} = \underline{D} \cdot \underline{S}_a \times \underline{S}_b.
    \label{AnisotropicInteractionDM}
\end{equation}

\noindent In this case, the energy of the system is minimised when magnetic moments are perpendicular. If this interaction is present in an anti-ferromagnetic system then the system becomes weakly ferromagnetic. This interaction is anisotropic because it occurs between the ground state of one magnetic ion and the excited state of another.

\noindent Most materials possess some magnetic anisotropy. This means it is energetically favourable for the material to be magnetised along a particular direction, which normally arises due to the crystal structure of the material. This means:

\begin{equation}
    E \propto \Sigma_i (S^Z_i)^2,
    \label{MagneticAnisotropyEnergy}
\end{equation}

\noindent where $D$ represents some characteristic property of the system providing anisotropy. If $D < 0$ then we have whats known as the easy axis along the z axis, whilst if $D > 0$ then the easy plane corresponds to the xy plane. This concept leads to the Ising Hamiltonian:

\begin{equation}
    H = -\Sigma_{i,j} J_{ij} S^Z_i S^Z_j,
    \label{IsingHamiltonian}
\end{equation}

\noindent and the XY hamiltonian:

\begin{equation}
    H = -\Sigma_{i,j} J_{ij} (S^X_i S^X_j + S^Y_i S^Y_j),
    \label{HamiltonianXY}
\end{equation}

\noindent respectively.

\section{Magnetic Measurement Techniques}

NEED IMAGES FOR EXPERIMENTAL SETUPS AND RESULTS.

\subsection{Magnetic Resonance}

The first technique is that of magnetic resonance. The experimental setup is shown below:

\begin{figure}
    \centering
    \includegraphics{}
    \caption{Caption}
    \label{fig:enter-label}
\end{figure}

\noindent Normally the frequency of microwave, $\nu$, is kept constant whilst the magnetic field is sweeped. The absorption of microwaves by the sample is determined by monitoring the Q factor of the cavity. In such magnetic dipole transitions the selection rules are $\Delta m_J = \pm1$ and $\Delta m_I = 0$.

\noindent For a $J = 1$ ion, such as Ni$^{2+}$m with no crystal field splitting we observe:

\begin{figure}
    \centering
    \includegraphics{}
    \caption{Caption}
    \label{fig:enter-label}
\end{figure}

\noindent However, with crystal field splitting:

\begin{figure}
    \centering
    \includegraphics{}
    \caption{Caption}
    \label{fig:enter-label}
\end{figure}

\noindent This allows us to determine the crystal field splitting, $\Delta$. A more complete picture, for Ce$^{3+}$, is shown below:

\begin{figure}
    \centering
    \includegraphics{}
    \caption{Caption}
    \label{fig:enter-label}
\end{figure}

\subsection{Magnetic Susceptibility Measurements}

We now consider two techniques that allow magnetic susceptibility measurements. The first is that of a torque magnetometer, as shown below:

\begin{figure}
    \centering
    \includegraphics{}
    \caption{Caption}
    \label{fig:enter-label}
\end{figure}

\noindent A torque, $\underline{\tau} = \underline{\mu} \times \underline{B}$, is produced on the sample using the solenoids which causes the wire to rotate. This rotation is detected by the photodetectors which then supply a current to the compensation coil which produces an opposing torque using the permanent magnets of known strength. The current is varied in order to compensate for the torque of the sample. These magnetometers are very sensitive but don't give the absolute value for the torque/magnetisation.

\noindent These can be measured exactly using extraction magnetometers:

\begin{figure}
    \centering
    \includegraphics{}
    \caption{Caption}
    \label{fig:enter-label}
\end{figure}

\noindent Here a magnetic material oscillates between a coil of wire, which through Lenz's law produces a measurable voltage:

\begin{equation}
    \epsilon = - \frac{d \Phi}{dt}
    \label{LenzLaw}
\end{equation}

\noindent This is the basic process behind vibrating sample magnetometers (VSMs) and superconducting quantum interference devices (SQUIDs).

\subsection{Calorimetry}

These type of measurements allow the determination of the heat capacity of a material which possesses knowledge of the interactions that have determined its properties. It also allows for the investigation of phase transitions.

\noindent The magnetic entropy of a system can be estimated from its heat capacity:

\begin{equation}
    S(T_f) = \int_0^{T_f} \frac{C(T)}{T} dT,
    \label{EstimatedMagneticEntropy}
\end{equation}

\noindent and then compared to the expected entropy of a magnetic ion with total angular momentum $J$:

\begin{equation}
    S = Rln(2J + 1)
    \label{ExactMagneticEntropy}.
\end{equation}

\noindent The principal method for determining the specific heat capacity is that of thermal relaxation. This measures the temperature of a material as a function of time and heater power. From the time constant associated with the heating and cooling the specific heat capacity can be determined.

\subsection{Magnetic Neutron Scattering}

This is typically performed using a triple-axis spectrometer. The monochromator crystal ensures that the neutrons have a single wavelength and not a distribution as they would have from a source.

\noindent The neutron has some properties that make it useful:

\begin{itemize}
    \item Similar wavelength to atomic spacings which means that diffraction measurements may be performed.
    \item Similar energy to atomic/electronic processes which allows for spectroscopy measurements.
    \item They are highly penetrating and so can be used for bulk measurements.
    \item They are non-destructive (when slow enough) so can be used to probe delicate samples.
\end{itemize}

\noindent Elastic magnetic scattering probes static magnetic structures whilst inelastic scattering probes the spin dynamics. The magnetic scattering cross section is proportional to the fourier transform of the spin correlation function.

\subsection{M\"ossbauer Spectroscopy}

This relies on the radioactive decay of $57^$Co to an excited state of $57^$Fe.

\begin{figure}
    \centering
    \includegraphics{}
    \caption{Caption}
    \label{fig:enter-label}
\end{figure}

\noindent The gamma ray emitted from the $57^$Fe between $I = \frac{3}{2}$ and $I = \frac{1}{2}$ has an energy of 14.4~keV. It is then directed towards a sample containing $57^$Fe. The doppler effect is present as the $57^$Co source moves towards the sample. The resulting absorption of photons is detected as a function of the speed fo the $57^$Co source. There is no recoil affecting the relative velocity due to the source being a bulk material. There is also a small probability of emission/absorption without a phonon. This process can be used to investigate quadrapole or magnetic splitting. However this process only works for certain materials.

\subsection{Muon-Spin Rotation}

Muons are prepared by colliding high energy protons with carbon nuclei which produces pions which in turn decay:

\begin{equation}
    \pi^+ \rightarrow \mu^+ + \nu_\mu
    \label{PionDecay}
\end{equation}

\noindent These muons emerge spin polarised and are implanted into the sample. Muons have a relatively long lifetime but eventually decay into positrons:

\begin{equation}
    \mu^+ \rightarrow e^+ + \Bar{\nu}_\mu + \nu_e
    \label{MuonDecay}
\end{equation}

\noindent These positrons tend to leave the sample along the direction which the muon spins was aligned along when the muon decayed. Typically this follows the distribution:

\begin{equation}
    W(\theta) = 1 + \alpha cos(\theta)
    \label{PositronDistribution}
\end{equation}

\noindent This method probes the local magnetic field of where the muon was implanted.

\section{Magnetic Domains}

When ferromagnets are cooled to $T < T_C$, magnetic domains are formed. These are regions of uniform magnetisation and are separated by domain walls of finite size.
The main domain walls are N\'eel walls and Bloch walls. In N\'eel walls the plane of rotation of magnetic moments is perpendicular to the plane of the wall, whilst for Bloch walls it is parallel.

\begin{figure}
    \centering
    \includegraphics{}
    \caption{Caption}
    \label{fig:enter-label}
\end{figure}

\noindent We now explore the Bloch wall further. The exchange energy of two spins in the wall is:

\begin{equation}
    E = -2J\underline{S}_1 \cdot \underline{S}_2 = -2JS^2\cos(\theta) \approx -2JS^2(1 - \frac{\theta^2}{2})
    \label{ExchangeEnergy2SpinBlochWall}
\end{equation}

\noindent This means that the energy cost to rotate one spin relative to another is given by:

\begin{equation}
    JS^2\theta^2
    \label{EnergyCostRotationBlochWall}
\end{equation}

\noindent So the energy cost for a rotation by $\pi$ with $N$ spins is, using Eq~\ref{EnergyCostRotationBlochWall}, given by:

\begin{equation}
    JS^2(\frac{\pi}{N})^2 N = \frac{JS^2\pi^2}{N},
    \label{FullEnergyCostBlochWall}
\end{equation}

\noindent which means the energy per unit area of a Bloch wall, $\sigma_{BW}$, in a cubic crystal is given by:

\begin{equation}
    \sigma_{BW} =  \frac{JS^2\pi^2}{Na^2},
    \label{EnergyPerUnitAreaBlochWall}
\end{equation}

\noindent where $a$ is the lattice parameter. However, this implies that $\sigma_{BW} \rightarrow 0$ as $N \rightarrow \infty$, meaning no domain walls would form. The interaction that prevents this and allows the formation of domain walls is due to magnetocrystalline anisotropy.

\noindent The energy cost comes from when the spins in the domain wall rotate. This means that they have a component along the hard plane/axis which in turn costs energy. We can model this magnetically preferred direction with a magnetic anisotropy energy given by:

\begin{equation}
    E = D\sin^2(\theta),
    \label{AnisotropyEnergy}
\end{equation}

\noindent where $D > 0$ is the anisotropy constant. So summing over all magnetic moments:

\begin{equation}
    \Sigma_{i = 1}^N D\sin^2(\theta) \approx \frac{N}{\pi} \int_0^\pi D\sin^2(\theta) d\theta = \frac{ND}{2}
    \label{TotalAnisotropyEnergyDensity}
\end{equation}

\noindent This means that the Bloch wall energy per unit area, combining Eq~\ref{EnergyPerUnitAreaBlochWall} and Eq~\ref{TotalAnisotropyEnergyDensity}, is given by:

\begin{equation}
    \sigma_{BW} = \frac{JS^2\pi^2}{Na^2} + \frac{NDa}{2}
    \label{FinalEnergyUnitAreaBlochWallExpression}
\end{equation}

\noindent From which we can find the minimum number of spins comprising the wall:

\begin{equation}
    \frac{d \sigma_{BW}}{dN} = 0 \Rightarrow N = \pi S \sqrt{\frac{2J}{Da^3}
    }
    \label{NumberSpinsBlochWall}
\end{equation}

\noindent This in turn means that the width of the wall is:

\begin{equation}
    \delta = Na = \pi S \sqrt{\frac{2J}{Da}},
    \label{BlochWallWidth}
\end{equation}

\noindent and also that:

\begin{equation}
    \sigma_{BW} = \pi S \sqrt{\frac{2JD}{a}}
    \label{FinalEnergyUnitAreaBlochWall}
\end{equation}

\noindent The formation of domain walls helps save dipolar energy. This is because of the necessity to satisfy $\underline{\nabla} \cdot \underline{B}$. So whenever a system possesses a magnetisation it also has a demagnetising field since, using Eq~\ref{MagneticFluxDensity}, $\underline{\nabla} \cdot \underline{M} = - \underline{\nabla} \cdot \underline{H}$.

\noindent The (surface) demagnetisation energy is given by:

\begin{equation}
    E = \frac{-\mu_0}{2} \int_V \underline{M} \cdot \underline{H}_d d\tau
    \label{DemagnetisationEnergy}
\end{equation}

\noindent For an ellipsoidal shape, magnetised along one of its principal axes:

\begin{equation}
    \underline{H}_d = - N\underline{M},
    \label{DemagnetisingField}
\end{equation}

\noindent where $N$ is the demagnetising factor. Combining Eq~\ref{DemagnetisationEnergy} and Eq~\ref{DemagnetisingField} we find that:

\begin{equation}
    E = \frac{\mu_0 N M^2 V}{2}
    \label{TotalDemagnetisationEnergy}
\end{equation}

\noindent This dipolar energy may be saved by breaking the sameple into domains. However the number of domains formed is a balance between the cost of demagnetisation and the cost of domain walls.

\begin{figure}
    \centering
    \includegraphics{}
    \caption{Caption}
    \label{fig:enter-label}
\end{figure}

\noindent For $\underline{H}$ parallel to a long cylindrical rod $N = 0$, whilst $N = 1$ for $\underline{H}$ perpendicular to a flat plate and $N = \frac{1}{3}$ for a sphere. Although $N$ is actually a matrix (trace must equal 1?).

\noindent The magnetisation of a material as a function of applied field in the form of a hysteresis loop shows the behaviour due to domains.

\begin{figure}
    \centering
    \includegraphics{}
    \caption{Caption}
    \label{fig:enter-label}
\end{figure}

\noindent This behaviour is due to the domain walls becoming pinned on defects. The magnetisation rotation region is where the domains are rotated away from the easy axis/plane due to the strength of the magnetic field.

\noindent Hysteresis loops can distinguish between hard and soft magnets:

\begin{figure}
    \centering
    \includegraphics{}
    \caption{Caption}
    \label{fig:enter-label}
\end{figure}

\noindent Soft magnets have the following properties:

\begin{itemize}
    \item Small anisotropy energy.
    \item Domain walls are broad and easy to move
    \item Small coercive fields
\end{itemize}

\noindent whereas hard magnets:

\begin{itemize}
    \item Domains are narrow and hard to move due to large anisotropy energy.
    \item No spontaneous magnetisation.
\end{itemize}

Magnetic materials can undergo a process known as magnetostriction. This is where the shape of a magnetic material changes when magnetised. This occurs since the crystal can lower its anisotropy energy by deforming itself. This is dependent on strain.

\noindent We now consider single domain small magnetic particles:

\begin{figure}
    \centering
    \includegraphics{}
    \caption{Caption}
    \label{fig:enter-label}
\end{figure}

\noindent It is possible for small magnetic particles with a single domain to form.

\noindent The behaviour of such a particle in an external magnetic field can be investigated with:

\begin{equation}
    E = D\sin^2(\theta - \phi) - \mu_0 H M \cos(\phi),
    \label{SingleDomainMagneticParticleEnergy}
\end{equation}

\noindent where $\theta$ is the angle between $\underline{H}$ and the easy axis, and $\phi$ is the angle between $\underline{M}$ and $\underline{H}$. If $\theta = 90^\circ$ we get a hysteresis loop for a soft magnet with plateaus at extreme magnetic fields and if $\theta = 0^\circ$ we obtain the hysteresis loop for a hard magnet.

\section{Paramagnets and Ferromagnets}

We first consider the semi-classical approach to determine the properties of a paramagnet. From Eq~\ref{EnergyMagneticMoment} we know that $E = -\underline{\mu} \cdot \underline{B} = -\mu B \cos(\theta)$. So the magnetic moments can orientate themselves anywhere in the interval [0, $\pi$).
We can determine the behaviour of the paramagnetic system using the continuous partition function:

\begin{equation}
    Z = \int^\pi_{-\pi} e^{-\mu B \beta \cos(\theta)} d\theta.
    \label{SemiClassicalPartitionFunction}
\end{equation}

\noindent Then by using:

\begin{equation}
    m = - \frac{\partial F}{\partial B},
    \label{MagneticMomentPerSpin}
\end{equation}

\noindent where $F = -k_B T lnZ$ is the Helmholtz free energy and $m$ is the magnetic moment per spin, and by using $M = nm$ where $n$ is the number density of spins, we find:

\begin{equation}
    \frac{M}{M_{SAT}} = \coth(y) - \frac{1}{y} \equiv L(y),
    \label{SemiClassicalMagnetisation}
\end{equation}

\noindent where $L(y)$ is the Langevin function and:

\begin{equation}
    y = \frac{\mu B}{k_B T}
    \label{SemiClassicalY}
\end{equation}

\noindent For small y, i.e small $\frac{B}{T}$, the Langevin function approximates to:

\begin{equation}
    L(y) \approx \frac{y}{3} = \frac{\mu B}{3 k_B T} = \frac{M}{M_{SAT}},
    \label{ApproximatedSemiClassicalMagnetisation}
\end{equation}

\noindent since for small y, $\coth(y) \approx \frac{1}{y} + \frac{y}{3}$. From Eq~\ref{ApproximatedSemiClassicalMagnetisation} we can determine the susceptibility:

\begin{equation}
    \chi = \frac{dM}{dH} = \frac{n \mu_0 \mu^2}{3 k_B T} = \frac{C}{T},
    \label{SemiClassicalySusceptibility}
\end{equation}

\noindent which is known as the Curie Law and we have used:

\begin{equation}
    M_{SAT} = n \mu,
    \label{SemiClassicalSaturatedMagnetisation}
\end{equation}

\noindent and that $B \approx \mu_0 H$.

\noindent We now perform the same calculation but for a quantum mechanical system. This means the magnetic moments can only occupy certain quantised values. For a $J=\frac{1}{2}$ paramagnet (where $L=0$ and $S=\frac{1}{2}$) the energy levels within a magnetic field are given by:

\begin{equation}
    E = g m_j \mu_B B = \pm \mu_B B,
    \label{EnergyLevelsHalfParamagnet}
\end{equation}

\noindent as $g\approx2$ and $m_j = \pm\frac{1}{2}$. Then following the same process as before by using the discrete partition function and letting:

\begin{equation}
    y = \frac{\mu_B B}{k_B T},
    \label{QuantumY}
\end{equation}

\noindent we find that:

\begin{equation}
    \frac{M}{M_{SAT}} = \tanh(y),
    \label{QuantumMagnetisation}
\end{equation}

\noindent where:

\begin{equation}
    M_{SAT} = n \mu_B.
    \label{QuantumSaturatedMagnetisation}
\end{equation}

\noindent This means for small $\frac{B}{T}$ we find that the susceptibility takes the form:

\begin{equation}
    \chi = \frac{n \mu_0 \mu_B^2}{k_B T} = \frac{C}{T},
    \label{QuantumSusceptibility}
\end{equation}

\noindent which is another form of the Curie Law as it shows the same temperature dependence as for the semi-classical case.

\noindent This process can be repeated a final time for a paramagnet of arbitrary $J$, which now has energy levels given by:

\begin{equation}
    E = m_j g_j \mu_B B,
    \label{EnergyLevelsArbitraryParamagnet}
\end{equation}

\noindent where $g_j$ is the Lande g factor. Now letting:

\begin{equation}
    y = \frac{g_j \mu_B B J}{k_B T},
    \label{ArbitraryQuantumY}
\end{equation}

\noindent we find that:

\begin{equation}
    \frac{M}{M_{SAT}} = B_J(y) = \frac{2J + 1}{2J}\coth(\frac{2J + 1}{2J}y) - \frac{1}{2J}\coth(\frac{y}{2J}),
    \label{ArbitraryQuantumMagnetisation}
\end{equation}

\noindent where $B_J(y)$ is the Brillouin function. For small $\frac{B}{T}$, we again find the same temperatude dependence:

\begin{equation}
    \chi = \frac{n \mu_0 \mu_{eff}^2}{3 k_B T} = \frac{C}{T},
    \label{ArbitraryQuantumSusceptibility}
\end{equation}

\noindent where $\mu_{eff} = g_j \mu_B \sqrt{J(J + 1)}$. From Eq~\ref{ArbitraryQuantumMagnetisation} it is possible to see that $B_{\frac{1}{2}}(y) = \tanh(y)$ and $B_{\inf}(y) = L(y)$ as we would expect.

\noindent Next we develop the Weiss mean field model of ferromagnetism. The Hamiltonian for this system is a combination of an exchange term and a Zeeman term:

\begin{equation}
    H = \Sigma_{i,j} J_{ij} \underline{S}_i \cdot \underline{S}_j + g \mu_B \Sigma_j \underline{S}_j \cdot \underline{B}.
    \label{FerromagneticHamiltonian}
\end{equation}

\noindent The mean field approximation constitutes defining an effective molecular field that is produced by all sites in the system other than the $i^{th}$ site which it acts on:

\begin{equation}
    \underline{B}_{mf} = \frac{-2}{g \mu_B} \Sigma_j J_{ij} \underline{S}_j
    \label{MeanField}.
\end{equation}

\noindent This equation can be used to replace the exchange term in our Hamiltonian:

\begin{equation}
    \Sigma_{i,j} J_{ij} \underline{S}_i \cdot \underline{S}_j = -2 \Sigma_{j} J_{ij} \underline{S}_i \cdot \underline{S}_j = -2 \underline{S}_i \cdot \Sigma_{i,j} J_{ij} \underline{S}_j = g \mu_B \underline{S}_i \cdot \underline{B}_mf.
\end{equation}

\noindent As a result Eq~\ref{FerromagneticHamiltonian} becomes:

\begin{equation}
    H = g \mu_B \Sigma_i \underline{S}_i \cdot (\underline{B} + \underline{B}_mf)
    \label{MeanFieldHamiltonian}
\end{equation}

\noindent which is in the same form as that of the paramagnetic case. We now further assume that:

\begin{equation}
    \underline{B}_{mf} = \lambda \underline{M},
    \label{MeanFieldAssumption}
\end{equation}

\noindent where $\lambda > 0$. This means we can use our equations for the abitrary $J$ paramagnet but in an effective field given by $\underline{B} + \lambda \underline{M}$. So the magnetisation of this ferromagnetic system has the same as for the paramagnetic case, Eq~\ref{SemiClassicalMagnetisation}:

\begin{equation}
    \frac{M}{M_{SAT}} = B_J(y)
    \label{FerromagneticMagnetisation},
\end{equation}

\noindent but now Eq~\ref{SemiClassicalY} takes the form:

\begin{equation}
    y = \frac{g_j \mu_B J (B + \lambda M)}{k_B T}
    \label{FerromagneticY}
\end{equation}

\noindent This pair of equations can be solved graphically. The intersection points represent the magnetisation values the system can take. In the case of $B = 0$, for $T > T_C$ there is only one solution of $M = 0$, whilst for $T \leq T_C$ there are three, one of which is $M = 0$ whilst the other two are $M \neq 0$.
However, if $B \neq 0$, there is always a solution with $M \neq 0$ regardless of the temperature of the system.

\begin{figure}
    \centering
    \includegraphics{}
    \caption{Caption}
    \label{fig:enter-label}
\end{figure}

\noindent The Curie (critical) temperature can be obtained from the case where $B = 0$ by finding where the gradients of Eq~\ref{FerromagneticMagnetisation} and Eq~\ref{FerromagneticY} are equal to each other for small y:

\begin{equation}
    T_C = \frac{g_j \mu_B (J + 1) \lambda M_{SAT}}{3 k_B}
    \label{CurieTemperature}
\end{equation}

\noindent If $T\gg T_C$ and we apply a small magnetic field, we can use the $y\ll1$ approximation in Eq~\ref{FerromagneticMagnetisation}:

\begin{equation}
    B_J(y) \rightarrow \frac{y (J + 1)}{3J}
\end{equation}

\noindent which means, using Eq~\ref{CurieTemperature}, we find:

\begin{equation}
    \frac{M}{M_{SAT}} = \frac{T_C}{\lambda M_{SAT}} (\frac{B + \lambda M}{T})
    \label{ApproximatedFerromagneticMagnetisation}
\end{equation}

\noindent This allows us to determine the Curie-Weiss law for the magnetic susceptibility:

\begin{equation}
    \chi = \frac{dM}{dH} = \frac{\mu_0 T_C}{\lambda (T - T_C)} = \frac{C}{T- T_C} \sim \frac{1}{T - T_C}
    \label{CurieWeissLaw}
\end{equation}

\noindent since from Eq~\ref{ApproximatedFerromagneticMagnetisation}:

\begin{equation}
    M = \frac{T_C B}{\lambda (T - T_C)}
\end{equation}

\section{Landau Theory}

Landau theory is a theoretical technique, based on mean field theory, which describes continuous (2nd order) phase transitions.

\noindent The basic assumption is that the free energy density, $f$ of a system may be expanded as a power series about the critical point. In general the free energy density will be a function of many different parameters including the order parameter, $\eta$. If $T > T_C$ then $\eta = 0$ and the system is in a high symmetry phase, whilst for $T < T_C$, $\eta \neq 0$ and the system is in a low symmetry phase.

\noindent For a magnetic system we choose $\eta = M$, i.e the magnetisation of the system. The relevant parameters for this system is the temperature, $T$, and the presence of an external magnetic field, $H$. This means our expansion takes the form:

\begin{equation}
    f(T, M) = f_0(T) + \alpha_2(T) M^2 + \alpha_(T) M^4 + ... ,
    \label{FreeEnergyDensityPowerExpansion}
\end{equation}

\noindent where $\underline{M} \cdot \underline{M} = M^2$ and there are no odd powers of $M$ since there is no energy difference between $\pm M$ states.

\noindent For a given $T$, the system is in equilibrium when:

\begin{equation}
    \frac{\partial f}{\partial m} = 0,
    \label{EquilibriumCondition}
\end{equation}

\noindent and in a minimum when:

\begin{equation}
    \frac{\partial^2 f}{\partial m^2} > 0
    \label{MinimumCondition}
\end{equation}

\noindent Using Eq~\ref{FreeEnergyDensityPowerExpansion} in Eq~\ref{EquilibriumCondition} and Eq~\ref{MinimumCondition} respectively we find:

\begin{equation}
    2M\alpha_2(T) + 4\alpha_4(T)M^3 = 0
    \label{MagneticEquilibriumCondition}
\end{equation}

\begin{equation}
    \alpha_2(T) + 6\alpha_4(T)M^2 > 0
    \label{MagneticMinimumCondition}
\end{equation}

\noindent For $T > T_C$, $M = 0$, which means that $\alpha_2(T) > 0$  using Eq~\ref{MagneticMinimumCondition}. For $T < T_C$, $M \neq 0$, which means that from Eq~\ref{MagneticEquilibriumCondition} we see that $\alpha_2(T) = -2M^2\alpha_4(T)$ which when substituted into Eq~\ref{MagneticMinimumCondition} results in $M^2\alpha_4(T) > 0$ which in turn means that $\alpha_2(T) < 0$.
This means that $\alpha_2(T)$ changes sign at $T = T_C$, so we may write:

\begin{equation}
    \alpha_2(T) = (T - T_C)\alpha_0,
    \label{Alpha2Coefficient}
\end{equation}

\noindent where $\alpha_0 > 0$. As such we find from Eq~\ref{MagneticEquilibriumCondition}:

\begin{equation}
    M^2 = \frac{\alpha_0(T_C - T)}{2\alpha_4},
    \label{MagnetisationEquilibriumCondition}
\end{equation}

\noindent which is only valid in the case of $T < T_C$ and if we assume that $\alpha_4(T) = \alpha_4$.

\noindent So for $T > T_C$, Eq~\ref{FreeEnergyDensityPowerExpansion} becomes:

\begin{equation}
    f(T, M) = f_0(T)
    \label{PowerExpansionAboveTC}
\end{equation}

\noindent and for $T < T_C$, using Eq~\ref{MagnetisationEquilibriumCondition} in Eq~\ref{FreeEnergyDensityPowerExpansion} we find:

\begin{equation}
    f(T, M) = f_0(T) - \frac{\alpha_0(T)^2 (T_C - T)^2}{4\alpha_4}
    \label{PowerExpansionBelowTC}
\end{equation}

\noindent This behaviour is shown more clearly in the following:

\begin{figure}
    \centering
    \includegraphics{}
    \caption{Caption}
    \label{fig:enter-label}
\end{figure}

\noindent A major limitation of Landau theory is that it ignores correlations and fluctuations in the order paramater which become increasingly important close to $T_C$.

\noindent We now continue with this magnetic system and calculate its entropy and specific heat capacity which can give indications of phase transitions:

\begin{equation}
    S = - \frac{\partial f}{\partial T}
    \label{Entropy}
\end{equation}

\begin{equation}
    C = T \frac{\partial S}{\partial T}
    \label{SpecificHeatCapacity}
\end{equation}

\noindent So for $T > T_C$, using Eq~\ref{PowerExpansionAboveTC}, we find:

\begin{equation}
    S = -\frac{\partial f_0}{\partial T} \Rightarrow C = -T\frac{\partial^2 f_0}{\partial T^2},
    \label{EntropySpecificHeatAboveTC}
\end{equation}

\noindent and using Eq~\ref{PowerExpansionBelowTC} for $T < T_C$:

\begin{equation}
    S = -\frac{\partial f_0}{\partial T} - \frac{\alpha_0^2(T_C - T)}{2\alpha_4} \Rightarrow C = -T\frac{\partial^2 f_0}{\partial T^2} + \frac{\alpha_0^2 T}{2\alpha_4}.
    \label{EntropySpecificHeatBelowTC}
\end{equation}

\noindent From this it is clear that there is a discontinuity in the specific heat capacity at $T = T_C$ of $\frac{\alpha_0^2 T_C}{2\alpha_4}$, as shown in the figure below:

\begin{figure}
    \centering
    \includegraphics{}
    \caption{Caption}
    \label{fig:enter-label}
\end{figure}

\noindent In reality a lambda anomaly is experimentally observed.

\noindent We now consider the same system but with a small extern magnetic field applied such that Eq~\ref{FreeEnergyDensityPowerExpansion} becomes:

\begin{equation}
    f = f_0(T) + \alpha_0 (T - T_C) M^2 + \alpha_4 M^4 - \mu_0 H M
    \label{FreeEnergyDensityPowerExpansionExternalField}
\end{equation}

\noindent Using the equilibrium condition on Eq~\ref{FreeEnergyDensityPowerExpansionExternalField} we find:

\begin{equation}
    \frac{\partial f}{\partial m} = 0 = 2\alpha_0 (T - T_C)M + 4\alpha_4 M^3 - \mu_0 H.
    \label{EquilibriumConditionExternalField}
\end{equation}

\noindent Then neglecting the cubic term as $M \ll 1$, we can rearrange the above expression:

\begin{equation}
    M = \frac{\mu_0 H}{2\alpha_0 (T - T_C)},
    \label{MagnetisationExternalField}
\end{equation}

\noindent such that the susceptibility of this system is found to take the form:

\begin{equation}
    \chi = \frac{dM}{dH} = \frac{\mu_0}{2\alpha_0(T - T_C)},
    \label{SusceptibilityExternalField}
\end{equation}

\noindent which has the same form as the Curie-Weiss Law, Eq~\ref{CurieWeissLaw}.

\noindent Since we have ignored correlations and fluctuations, as is typical in mean field theories, the values of critical exponents don't match the experimentally determined values. So instead of:

\begin{equation}
    M \sim (T_C - T)^{\frac{1}{2}},
    \label{CriticalExponentMagnetisationMeanField}
\end{equation}

\noindent we have:

\begin{equation}
    M \sim (T_C - T)^{\beta},
    \label{CriticalExponentMagnetisation}
\end{equation}

\noindent where $\beta \neq \frac{1}{2}$ and is thought to be an irrational number.

\noindent The critical exponent depends on the dimensionality of the system, $d$, the symmetry/dimensionality of the order parameter, $D$, and the range of the interaction. 

\section{Anti-Ferromagnets}

The Hamiltonian for an antiferromagnetic system is the same as that of a ferromagnetic system:

\begin{equation}
    H = -\Sigma_{i,j} J_{ij} \underline{S}_i \cdot \underline{S}_j + g \mu_B \Sigma_i \underline{S}_i \cdot \underline{B},
    \label{AntiFerromagneticHamiltonian}
\end{equation}

\noindent where $J < 0$ for such an anti-ferromagnetic systems.

\noindent We model this system as a superposition of two inter-penetrating sub-lattices. This results in the unit cell being twice as large.
Since there are two sub-lattices of equal but opposite magnetisation:

\begin{equation}
    \underline{M} = \underline{M}_+ + \underline{M}_- = 0,
    \label{OverallMagnetisationAntiFerromagnet}
\end{equation}

\noindent we use the staggered magnetisation as an order parameter:

\begin{equation}
    \eta = \underline{M}_+ - \underline{M}_-.
    \label{StaggeredMagnetisation}
\end{equation}

\noindent Taking the mean field approach, each sub-lattice feels an effective field due to the others magnetisation:

\begin{equation}
    \underline{B}_+ = - |\lambda| \underline{M}_-
\end{equation}

\begin{equation}
    \underline{B}_- = - |\lambda| \underline{M}_+
\end{equation}

\noindent where $\lambda > 0$. This means the magnetisation on each sub-lattice is given by:

\begin{equation}
    M_{\pm} = M_{SAT} B_J(y) = M_{SAT} B_J(\frac{g_j \mu_B J(B - |\lambda|M_{\mp})}{k_B T}).
    \label{SubLatticeMagnetisation}
\end{equation}

\noindent Then if we assume that $M = M_+ = M_-$, then in zero field Eq~\ref{SubLatticeMagnetisation} becomes:

\begin{equation}
    M = M_{SAT} B_J(\frac{g_j \mu_B J |\lambda| M}{k_B T},
    \label{SubLatticeMagnetisationZeroField}
\end{equation}

\noindent from which, in a similar process to determining the Curie temperature, the critical temperature for anti-ferromagnets (Neel temperature) is given by:

\begin{equation}
    T_N = \frac{g_j \mu_B (J + 1) |\lambda| M_{SAT}}{3k_B} = \frac{n |\lambda| \mu_{eff}^2}{3k_B},
    \label{NeelTemperature}
\end{equation}

\noindent above which magnetic order disappears.

\noindent So in general, above their respective critical temperatures the magnetic susceptibility of magnetic systems follow:

\begin{equation}
    \chi_{FM} \sim \frac{1}{T - T_C}
    \label{FerromagneticSusceptibility}
\end{equation}

\begin{equation}
    \chi_{AFM} \sim \frac{1}{T + T_N}
\end{equation}

\noindent which may be summarised as:

\begin{equation}
    \chi \sim \frac{1}{T - \theta},
    \label{MagneticSusceptiblityAboveCriticalTemperature}
\end{equation}

\noindent where $\theta$ is the Weiss constant. This can be used to determine the magnetic ordering present in a material whilst that material is above the critical temperature. If $\theta = 0$ then the material behaves as a paramagnet, similar to the Curie Law, Eq~\ref{ArbitraryQuantumSusceptibility}. If $\theta > 0$, the system is ferromagnetic as described by Eq~\ref{CurieWeissLaw}. If $\theta < 0$, the system is anti-ferromagnetic.

\noindent Normally the experimentally determined values are far from those predicted by the mean field approach. This is because we have ignored the effect of a sub-lattice on itself. So in reality, we have:

\begin{equation}
    B_{\pm} = -|\lambda|M_{\mp} + \Gamma M_{\pm},
\end{equation}

\noindent instead of Eq 1.97 and Eq 1.98. This means that Eq~\ref{NeelTemperature} becomes:

\begin{equation}
    T_N = \frac{n(|\lambda| - \Gamma)}{3k_B} \mu_{eff}^2
    \label{ModifiedNeelTemperature}
\end{equation}

\noindent We now consider the behaviour of an antiferromagnet for $T < T_N$. This is more difficult than the ferromagnetic case since the direction along which an external magnetic field is applied can cause differing behaviour.

\noindent We first consider small magnetic fields. Applying a magnetic field perpendicular:

\begin{figure}
    \centering
    \includegraphics{}
    \caption{Caption}
    \label{fig:enter-label}
\end{figure}

\noindent The energy of this two spin system is, as per usual, a combination of an exchange term and a zeeman term:

\begin{equation}
    E = U \underline{M}_A \cdot \underline{M}_B - \underline{B} \cdot (\underline{M}_A + \underline{M}_B)
    \label{TwoSpinAntiferromagnetEnergy}
\end{equation}

\noindent and since $\underline{B}$ is small, $\phi$ the angle between the spins original direction and that of when the magnetic field is applied will also be small, we can use the small angle approximations for $\sin(\theta)$ and $\cos(\theta)$, such that Eq~\ref{TwoSpinAntiferromagnetEnergy} becomes:

\begin{equation}
    E = UM^2[-1 + \frac{(2\phi)^2}{2}] - 2BM\phi
    \label{TwoSpinAntiferromagnetcEnergyApproximate}
\end{equation}

\noindent SHOULD ADD MORE DETAIL HERE ABOUT HOW WE GET THIS.

\noindent Minimising this expression with respect to the angle we find:

\begin{equation}
    \frac{\partial E}{\partial \phi} = 4UM^2\phi - 2BM = 0
    \label{MinimsedTwoSpinAntiferromagnetEnergy}
\end{equation}

\noindent Resulting in:

\begin{equation}
    \phi  = \frac{B}{2UM}
    \label{MinimsedAngleTwoSpinAntiferromagnet}
\end{equation}

\noindent The horizontal component of the total magnetisation is still zero, however there is now a vertical component given by:

\begin{equation}
    M_{TOT} = 2M\sin(\phi) \approx 2M\phi = \frac{B}{U},
    \label{TotalVerticalMagnetisationTwoSpinAntiferromagnet}
\end{equation}

\noindent using Eq~\ref{MinimsedAngleTwoSpinAntiferromagnet}. As a result we find that the susceptibility of this antiferromagnet to fields perpendicular to its magnetisation is given by:

\begin{equation}
    \chi = \frac{dM}{dH} = \frac{\mu_0}{U} = \chi_\perp,
    \label{LowFieldPerpendicularSusceptibility}
\end{equation}

\noindent which is a constant. If instead we apply the external field to be parallel to the magnetisation, such that for low fields Eq~\ref{OverallMagnetisationAntiFerromagnet} still holds, then for $T = 0$ we find $\chi_\parallel = 0$. This leads to the following behaviour for small external magnetic fields applied:

\begin{figure}
    \centering
    \includegraphics{}
    \caption{Caption}
    \label{fig:enter-label}
\end{figure}

\noindent Note that for a powder of antiferromagnetic material:

\begin{equation}
    \chi_{powder} = \frac{2}{3} \chi_\perp + \frac{1}{3} \chi_{\parallel}.
\end{equation}

\noindent We now consider the behaviour of antiferromagnets in high magnetic fields but it is still the case that $T < T_N$. Here we now account for the magnetic anisotropy that will exist in materials, so that Eq~\ref{TwoSpinAntiferromagnetEnergy} now becomes:

\begin{equation}
    E = U \underline{M}_A \cdot \underline{M}_B - \underline{B} \cdot (\underline{M}_A + \underline{M}_B) - |D|[(M_A^Z)^2 + (M_B^Z)^2],
    \label{TwoSpinAntiferromagnetAnisotropyEnergy}
\end{equation}

\noindent where $D < 0$, so the spins like to align along the easy axis, in this case the z axis. We further assume that the exchange energy dominates over the anisotropy energy such that $U \gg |D|$. This means the system prefers to try to keep the spins anti-aligned more than it prefers to keep them aligned along the easy axis.

\noindent For $B \perp z$ as we have already seen, from Eq~\ref{LowFieldPerpendicularSusceptibility}, $\chi_\perp$ is a constant so $M \sim B$. However the field acts against both the exchange and anisotropy energy so:

\begin{equation}
    M \sim \frac{B}{U + |D|} \approx \frac{B}{U},
    \label{HighFieldPerpendicularSusceptibility}
\end{equation}

\noindent which returns us the same behaviour as before. The more interesting behaviour is observed for $B \parallel z$. In the low field case $\chi_\parallel = 0$ at $T = 0$ since $M_{TOT} = 0$. However in the high field case a spin flop transition occurs when $B > B_{SF}$.

\begin{figure}
    \centering
    \includegraphics{}
    \caption{Caption}
    \label{fig:enter-label}
\end{figure}

\noindent In this case the magnetic field is competing with the exchange energy but working with the anisotropy energy such that:

\begin{equation}
    M \sim \frac{B}{U - |D|} \approx \frac{B}{U},
    \label{HighFieldParallelSusceptibility}
\end{equation}

\noindent from which it can be seen that $\chi_\parallel \approx \chi_\perp$ for $B > B_{SF}$. This spin flop transition is a 1st order phase transition.

\noindent This behaviour for $B \parallel z$ is summarised in the magnetic phase diagram:

\begin{figure}
    \centering
    \includegraphics{}
    \caption{Caption}
    \label{fig:enter-label}
\end{figure}

\noindent The main characteristics of antiferromagnets to remember are that since the exchange constant, J, is negative it favours anti-parallel alignments of spins. This results in systems which have no net magnetisation.

\noindent Note that in antiferromagnets you can have spin flip transitions but this only occurs when the anisotropy energy has a similar or larger contribution than the exchange energy. This ensures that the spins have a very strong tendency to remain along the easy axis.

\section{More Magnetic Order Types}

Ferrimagnetic materials are in essence the same as antiferromagnets, however their net magnetisation is non-zero. They are usually compounds containing different magnetic ions, or mixed valency ions or the same magnetic ion in two different positions.

\noindent The molecular field on each sub-lattice will now be different resulting in more complex temperature dependence. One sub-lattice might dominate at high temperatures whilst the other dominates at low temperatures. This makes the temperature dependence of the magnetic susceptibility more involved and also means the Curie-Weiss law is no longer applicable. There is however a temperature at which the overall magnetisation is 0, this is known as the compensation temperature.

\noindent Another magnetic order type is that of helical order (helimagnetism). In rare earth metals the atoms lie in layers. So within the layer there is ferromagnetic coupling, but between nearest neighbour layers there is an exchange constant, $J_1$, and between next-nearest neighbour layers an exchange constant, $J_2$.

\begin{figure}
    \centering
    \includegraphics{}
    \caption{Caption}
    \label{fig:enter-label}
\end{figure}

\noindent Here $\theta$ is the angle between spins in successive layers. This can result in magnetic structure which is not periodic with the crystal structure.]

\noindent The energy of three layers is given by:

\begin{equation}
    E = -NS^2[J_1 \cos(\theta) + J_2 \cos(2\theta)],
    \label{HelicalOrderEnergy}
\end{equation}

\noindent where $N$ is the total number of spins in each plane. Minimising this:

\begin{equation}
    \frac{\partial E}{\partial \theta} = 0 = J_1 \sin(\theta) + 2J_2 \sin(2\theta),
    \label{MinimisedHelicalOrderEnergy}
\end{equation}

\noindent which can be rearranged to:

\begin{equation}
    [J_1 + 4J_2\cos(\theta)] \sin(\theta) = 0.
    \label{RearrangedMinimisedHelicalOrderEnergy}
\end{equation}

\noindent This has two sets of solutions. Either $\sin(\theta) = 0$, so $\theta = 0$ or $\theta = n\pi$, which are either ferromagnetic  or antiferromagnetic behaviour; or:

\begin{equation}
    \cos(\theta) = \frac{-J_1}{4J_2},
    \label{HelicalOrderCondition}
\end{equation}

\noindent which corresponds to helical order. In order for this configuration to be stable we need $J_2 < 0$ and $|J_1| < 4|J_2|$.

\begin{figure}
    \centering
    \includegraphics{}
    \caption{Caption}
    \label{fig:enter-label}
\end{figure}

\noindent Another magnetic structure which doesn't necessarily match the periodicity of the crystal structure, i.e is incommensurate with it, is the spin density wave. This is formed from collinear spins producing antiferromagnetic behaviour, where the lengths of magnetic moments vary with time.

\begin{figure}
    \centering
    \includegraphics{}
    \caption{Caption}
    \label{fig:enter-label}
\end{figure}

\noindent The final magnetic structure is that of spin glasses. These are made by taking a non-magnetic lattice and populating it sparsely and randomly with magnetic ions. This means there is site and bond randomness, such that the strength of the magnetic interactions is random. The final criteria for such a spin glass is for it to possess magnetic frustration.

\noindent Spin glasses have a freezing temperature, $T_f$, below which there is cooperative freezing of the spins, so they are frozen in the state they occupied when the spin glass reached $T_f$.

\noindent For $T > T_f$ and $T < T_c$, above which it is paramagnetic, the system forms cluster states where independent spins slow down and are only locally connected. As the material approaches $T_f$ there are fewer fluctuations and the interactions become more long range.

\noindent For $T < T_f$ there is no long range magnetic order and thus no Bragg peaks from neutron scattering. The system now occupies one of many possible degenerate states. This is characterised by a very slow relaxation time and having a history dependence. The material is effectively frozen.

\noindent These spin glass states can be characterised using the real and imaginary parts of $\chi$ which is determined using an alternating magnetic field so $\chi = \chi(\omega)$. 

\section{Magnetic Excitations}

We shall take the semiclassical approach to spin waves in ferromagnets, which are quantised as magnons.

\noindent Consider a 1D chain of $N$ spins, separated by $a$, and coupled by a ferromagnetic interaction:

\begin{equation}
    H = -2J \Sigma_{i = 1}^N \underline{S}_i \cdot \underline{S}_{i + 1} = E_0
    \label{SpinWaveHamiltonian}
\end{equation}

\noindent In the classical ground state each spin is parallel since $J > 0$:

\begin{figure}
    \centering
    \includegraphics{}
    \caption{Caption}
    \label{fig:enter-label}
\end{figure}

\noindent One possible excitation is to reverse one spin:

\begin{figure}
    \centering
    \includegraphics{}
    \caption{Caption}
    \label{fig:enter-label}
\end{figure}

\noindent where the new energy of the spin is now:

\begin{equation}
    E = E_0 + 8JS^2
    \label{ReversedSpinEnergy}
\end{equation}

\noindent since for the $p^${th} spin:

\begin{equation}
    E = -2J(\underline{s}_{p - 1} \cdot \underline{s}_p + \underline{s}_{p} \cdot \underline{s}_{p + 1})
    \label{PthSpinEnergy}
\end{equation}

\noindent which shows that by reversing a spin you remove $4JS^2$ of energy and then add another $4JS^2$ worth of energy.

\noindent Another possible excitation, which represents a lower energy state, is the spin wave. In this case the spins share the spin flip amongst themselves.

\noindent We can also treat this with a mean field theory by introducing a magnetic field $\underline{B}_p$ which affects the $p^${th} spin and arises due to its nearest neighbours:

\begin{equation}
    E = -\underline{\mu}_p \cdot \underline{B_p} = g \mu_B \underline{S}_p \cdot [\frac{-2J}{g \mu_B}(\underline{S}_{p - 1} + \underline{S}_{p + 1})]
    \label{MeanFieldTheorySpinWaveEnergy}
\end{equation}


\noindent where $\underline{\mu_p}$ is the magnetic moment on the $p^${th} site which is affected by $\underline{B}_p$:

\begin{equation}
    \underline{\mu}_p = -g \mu_B \underline{S}_p
    \label{MagneticMomentPthSite}
\end{equation}

\begin{equation}
    \underline{B}_p = \frac{-2J}{g\mu_B} (\underline{S}_{p - 1} + \underline{S}_{p + 1})
    \label{MagneticFieldPthSite}
\end{equation}

\noindent The resulting torque on the spin on the $p^{th}$ site is given by:

\begin{equation}
    \underline{\tau}_p = \mu_p \times \underline{B}_p
    \label{TorquePthSite}
\end{equation}

\noindent which can be expressed as:

\begin{equation}
    \hbar \frac{d\underline{S}_p}{dt} = \underline{\tau}_p = 2J[\underline{s}_p \times \underline{s}_{p + 1} + \underline{s}_p \times \underline{s}_{p - 1}],
    \label{ProperTorquePthSite}
\end{equation}

\noindent where we have used the fact that torque is the rate of change of angular momentum and in this case the only contribution to the angular momentum is through spin, so there is no orbital contribution.

\noindent Expanding Eq~\ref{ProperTorquePthSite} into its cartesian coordinates we obtain three coupled non-linear ODEs. However they may be linearised by assuming deviations from the ground state are small such that:

\begin{equation}
    S_p^x, S_p^y \ll S = S_p^z
    \label{LinearisingConditionSpinWaves}
\end{equation}

\noindent As a result of this the equations of motion become:

\begin{eqnarray}
    \frac{dS_p^x}{dt} = \frac{2JS}{\hbar}(2S_p^y - S_{p - 1}^y - S_{p + 1}^y) \\
    \frac{dS_p^y}{dt} = \frac{2JS}{\hbar}(2S_p^x - S_{p - 1}^x - S_{p + 1}^x) \\
    \frac{dS_p^z}{dt} = 0 \hspace{37.5mm}
\end{eqnarray}

\noindent We now assume that the solutions of these equations are travelling wave:

\begin{eqnarray}
    S_p^x = u e^{i(pka - \omega t)} \label{AssumedSpinWaveSolution1}\\ 
    S_p^y = v e^{i(pka - \omega t)} \label{AssumedSpinWaveSolution2}
\end{eqnarray}

\noindent where $u$ and $v$ are the amplitudes. By substituting these assumed solutions into our equations of motion we find they obey a matrix equation which is solved by ensuring that the determinant of the matrix is zero. This results in the corresponding dispersion for the spin waves:

\begin{equation}
    \hbar \omega = 4JS[1 - \cos(ka)] \approx 2JSa^2k^2
    \label{SpinWavesDispersionRelation}
\end{equation}

\noindent where we have assumed that the wavelengths of these spin waves are very long such that $ka \ll 1$. We also find that our assumed solutions, Eq~\ref{AssumedSpinWaveSolution1} and Eq~\ref{AssumedSpinWaveSolution2} are out of phase with each other by $\frac{\pi}{2}$, i.e $v = -iu$, which means:

\begin{eqnarray}
    S_p^x = u\cos(pka - \omega t) \\.
    S_p^y = v\sin(pka - \omega t)
\end{eqnarray}

\noindent This is very similar to phonons except there dispersion relationship is linear in $k$ whereas for spin waves it is quadratic.


If instead the previous calculation was performed using quantum mechanics we would find that the dispersion relationship has the same form but now the energy of such modes is quantised according to:

\begin{equation}
    E_k = (n_k + \frac{1}{2}) \hbar \omega_k
    \label{QuantisedModeSpinWave}
\end{equation}

\noindent The quanta associated with these spin waves are known as magnons. They correspond to flipping the spin of an electron so the magnetisation decreases by $g\mu_B$ and must be bosons since they flip the spin of an electron from $+\frac{1}{2}$ to $-\frac{1}{2}$ meaning they possess spin 1. This means they obey the Bose-Einstein statistics:

\begin{equation}
    n(\omega) = \frac{1}{e^{\frac{\hbar \omega}{k_B T}} - 1}
    \label{BoseEinsteinStatistics}
\end{equation}

\noindent In order to determine the number of magnons in a system at temperature $T$, to find out how the magnetisation varies with temperature, we need the density of states of electrons within the material

\begin{equation}
    g(k) dk = \frac{Vk^2}{2\pi^2} dk \propto k^2 dk
    \label{kDependenceDOS}
\end{equation}

\noindent and since, according to Eq~\ref{SpinWavesDispersionRelation}, $\omega \propto k^2$:

\begin{equation}
    g(\omega) d \omega \propto \sqrt{\omega} d\omega
    \label{omegaDependenceDOS}
\end{equation}

\noindent This means we find the number of magnons as:

\begin{equation}
    N = \int^\infty_0 \frac{g(\omega)}{n(\omega)} d\omega = (\frac{k_B T}{\hbar})^{\frac{3}{2}} \int^\infty_0 \frac{\sqrt{x}}{e^x - 1} dx \sim T^{\frac{3}{2}}
    \label{NumberMagnons}
\end{equation}

\noindent where $x = \frac{\hbar \omega}{k_B T}$ and there are some shenanigans going on. From this we can determine Bloch's law:

\begin{equation}
    \frac{M(0) - M(T)}{M(0)} \sim T^{\frac{3}{2}}
    \label{Bloch3/2Law}
\end{equation}

\noindent which is only valid for small temperatures since we have assumed that $ka \\l 1$, which means that only low energy (large wavelength) magnon excitations are allowed.

\noindent This result is seen experimentally at low temperatures:

\begin{figure}
    \centering
    \includegraphics{}
    \caption{Caption}
    \label{fig:enter-label}
\end{figure}

\noindent This behaviour is not seen in the Ising model, since the spins are constrained to the easy axis (i.e can only point up or down), whilst the spin wave theory we have developed was formulated from the Heisenberg Hamiltonian.

%which says that the number of magnons in a system at a low temperature (low energy excitations leads to large wavelength spin waves) scales as $T^{\frac{3}{2}}$



